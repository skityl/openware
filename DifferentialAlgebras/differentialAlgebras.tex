\documentclass{owmaths}

\usepackage{owshortcuts}
\usepackage[all]{xy}

\begin{document}

\subject{Non-Commutative Geometry}
\author{Edward McDonald}
\title{Non-commutative Differential Algebras}
\studentno{616}

\newcommand{\Cliff}{\operatorname{Cliff}}
\newcommand{\im}{\operatorname{im}}
\newcommand{\com}{\mathrm{com}}
\newcommand{\Hom}{\operatorname{Hom}}
\newcommand{\A}{\mathcal{A}}
\newcommand{\Hilb}{\mathcal{H}}
\newcommand{\D}{\mathcal{D}}


\setlength\parindent{0pt}

\section{Introduction}
For several decades now, mathematicians have been attempting to find analogues
of theorems of differential and algebraic geometry in noncommutative 
algebra.
The biggest obstacle to learning this topic is that many of the definitions
were arrived at through many years of hard work, and may seem unmotivated at first.
The purpose of these notes is to explain the algebra of Connes Differentials,
which were invented by Alain Connes as a noncommutative generalisation
of the exterior algebra bundle on a manifold.

\section{Classical Differential Algebra}
Let $M$ be an $n$ dimensional manifold. The cotangent bundle $\Omega^1(M)$
is a rank $n$ vector bundle on $M$. We build higher bundles by wedge products,
\begin{equation*}
    \Omega^p(M) := \bigwedge_p \Omega^1(M).
\end{equation*}
and define $\Omega^0(M) := C^\infty(M)$. See that $\dim_\Rl(\Omega^p(M)) = \binom{n}{p}$.


The exterior algebra bundle is the direct sum of all the $\Omega^p(M)$,
\begin{equation*}
    \Omega(M) := \bigoplus_{p=0}^\infty \Omega^p(M).
\end{equation*}


$\Omega(M)$ is a \emph{graded algebra}.


In general, if $A$ is an algebra over a ring $R$, we say that $A$ is $\Ntrl$-graded
if there exists a decomposition into submodules $A^{(p)}$,
\begin{equation*}
    A = \bigoplus_{p=0}^\infty A^{(p)}
\end{equation*}
such that $A^{(n)}A^{(m)} \subseteq A^{(n+m)}$.

In the case $A = \Omega(M)$, we have $R = \Rl$, and $A^{(p)} = \Omega^p(M)$.

The exterior derivative, $d:\Omega(M)\rightarrow \Omega(M)$ acts on the grading by,
\begin{equation*}
    d:\Omega^p(M) \rightarrow \Omega^{p+1}(M).
\end{equation*}
and $d^2 = 0$. Hence we have a sequence,

\begin{equation*}
    0\rightarrow \Omega^{0}(M) \xrightarrow{d} \Omega^1(M) \xrightarrow{d} \cdots \xrightarrow{d} \Omega^n(M) \xrightarrow{d} 0.
\end{equation*}
Denote $d_p$ as the restriction of $d$ to $\Omega^p(M)$. Then we have the \emph{de Rham cohomology}
spaces,
\begin{equation*}
    H^p_{dR}(M) = \frac{\ker(d_{p})}{\im(d_{p-1})}
\end{equation*}
This is a sequence of real vector spaces, and their dimensions are topological
invariants of $M$.

The maps $d_p$ satisfy a graded version of Leibniz's rule, for $a \in \Omega^n(M)$
and $b \in \Omega^m(M)$, we have:
\begin{equation*}
    d_{n+m}(ab) = d_n(a)b+(-1)^nad_m(b)
\end{equation*}

\section{Abstract Differential Algebra}
\subsection{Graded Differential Algebras}
We now take the ideas of the previous section and move them to a more abstract setting.
Let $R$ be a commutative ring, and let $A$ be an $\Ntrl$-graded algebra over $R$, with decomposition
\begin{equation*}
    A = \bigoplus_{p=0}^\infty A^{(p)}
\end{equation*}

There is also an $R$-linear map $d:A\rightarrow A$ such that,
\begin{equation*}
    d:A^{(p)}\rightarrow A^{(p+1)}.
\end{equation*}
and $d^2 = 0$.
If we denote the restriction of $d$ to $A^{(p)}$ as $d_p$, we
require that the maps $d_p$ satisfy a graded Leibniz rule, for $a \in A^{(n)}$
and $b \in A^{(m)}$,
\begin{equation*}
    d_{n+m}(ab) = d_n(a)b+(-1)^nad_m(b).
\end{equation*}

A pair $(A,d)$ satisfying these conditions is called a \emph{differential graded algebra}.

Thus we have a sequence,
\begin{equation*}
    0\rightarrow A^{(0)} \xrightarrow{d} A^{(1)} \xrightarrow{d} \cdots \xrightarrow{d} A^{(n)} \xrightarrow{d} \cdots
\end{equation*}
The quotient $R$-modules,
\begin{equation*}
    H^p_{dR}(M) := \frac{\ker(d_{p})}{\im(d_{p-1})}.
\end{equation*}
are the de Rham cohomology modules for the graded differential algebra $(A,d)$.
\subsection{K\"ahler Differentials}
Given an $R$-algebra $A$, we would like to be able to build an algebra
of differential forms over $A$, in a manner analogous to how $\Omega^1(M)$
is constructed from $C^\infty(M)$. It turns out that there is a good way of doing
this, called the algebra of \emph{K\"ahler differentials}. This is simplest in the commutative case,
which we briefly outline here.

Let $R$ be a commutative ring, and let $A$ be a unital commutative $R$-algebra. The module
$\Omega^1_{\com}(A)$
of K\"ahler differentials is defined as
\begin{equation*}
    \Omega^1_{\com}(A) := \frac{A\otimes_R A}{\langle c\otimes(ab)-(ca)\otimes b-(bc)\otimes a\rangle}
\end{equation*}
The idea here is that $\Omega^1_{\com}(A)$ is the left $A$-module spanned by all
symbols of the form $adb$, where $d(ab) = adb+bda$. We think of $a\otimes b$
as $adb$.

More precisely, we let $d:A\rightarrow \Omega^1_{\com}(A)$ be given by
\begin{equation*}
    da := 1_A\otimes a
\end{equation*}
Where $1_A$ is the unit in $A$.

The utility of $\Omega^1_{\com}(A)$ is that it allows us to study all
derivations on $A$. 

In full abstraction, a derivation on $A$ is a map $\theta:A\rightarrow M$, where
$M$ is some left $A$-module, such that $\theta$ satisfies the leibniz rule,
\begin{equation*}
    \theta(ab) = a\theta(b)+b\theta(a).
\end{equation*}

We see that $d$ is a derivation on $A$ to the $A$-module $\Omega^1_{\com}(A)$.
It is in fact universal with this property,
\begin{theorem}
    Let $A$ be a unital commutative $R$-algebra, and let $\theta:A\rightarrow M$
    be a derivation to some left $A$-module $M$. There exists a unique $R$-linear 
    map $\Omega(\theta)$ such that the following diagram commutes,\\
    \begin{displaymath}
    \xymatrix{
        A \ar[r]^d \ar[rd]^\theta & 
        \Omega^1_{\com}(A) \ar@{.>}[d]^{\Omega(\theta)} &\\
         &
        M
  } 
  \end{displaymath}
  In other words, there is an isomorphism of $R$-modules,
  \begin{equation*}
    D(A,M) \cong \Hom_R(\Omega^1_{\com}(A),M).
  \end{equation*}
  Where $D(A,M)$ is the set of derivations from $A$ to $M$.
  Note that this universal property defines $\Omega^1_{\com}(A)$ up
  to unique isomorphism.
\end{theorem}
\begin{proof}
    Basically, $\Omega(\theta)$ maps $adb$ to $a\theta(b)$. Checking the universal property
    is routine.
\end{proof}

We would now like to create a similar algebra of differentials for a non-commutative
associative algebra $A$ over $R$. In the noncommutative case, we must restrict
attention to derivations that take values in $A$-bimodules, rather than left
$A$ modules.
\begin{definition}
    Let $A$ be an associative
    unital algebra over a commutative ring $R$. 
    Let $m:A\otimes A\rightarrow A$ be the multiplication map. We define
    \begin{equation*}
        \Omega^1(A) = \ker(m)
    \end{equation*}
    This is an $A$ bimodule.
\end{definition}
This the motivation behind this definition is not at all clear. However, this
does agree with the commutative case and this provides the appropriate definition
for noncommutative K\"ahler differentials. To see this, we define the map
$d:A\rightarrow \Omega^1(A)$, by
\begin{equation*}
    d(a) = 1_A\otimes a-a\otimes 1_A.
\end{equation*}
We see that $d$ is a derivation. In fact, $\Omega^1(A)$
should be thought of as the space of all linear combinations
of terms of the form $ad(b)$.


$\Omega^1(A)$ satisfies the 
same universal property as $\Omega^1_{\com}$. Namely, if $M$ is an $A$-bimodule,
and $\theta:A\rightarrow M$ is a derivation, then there exists a unique $R$-linear
map $\Omega(\theta)$ such that the following diagram commutes,

    \begin{displaymath}
    \xymatrix{
        A \ar[r]^d \ar[rd]^\theta & 
        \Omega^1(A) \ar@{.>}[d]^{\Omega(\theta)} &\\
         &
        M
  } 
  \end{displaymath}
  
\subsection{Universal Differential Algebra}
Given an associative unital algebra $R$ over a commutative ring $R$,
we define
\begin{equation*}
    \Omega^p(A) := \bigotimes_{A,p} \Omega^1(A).
\end{equation*} 
And the algebra,
\begin{equation*}
    \Omega A = \bigoplus_{p} \Omega^p(A).
\end{equation*}
We extend the function $d:A\rightarrow \Omega^1(A)$ to $\Omega A$
by
\begin{equation*}
    d(ada_1da_2\cdots da_n) = dada_1da_2\cdots da_n.
\end{equation*}
 
\begin{theorem}
    $\Omega A$ is the ``largest" graded differential algebra generated by $A$. 
    
    If $(\Gamma,\Delta)$ is a graded differential algebra, with grading $\Gamma = \bigoplus_n \Gamma^{(n)}$,
    and $\rho:A\rightarrow \Gamma^{(0)}$ is an algebra homomorphism, then $\rho$
    extends uniquely to a morphism $\Omega A\rightarrow \Gamma$ such that the following
    diagram commutes,
    \begin{displaymath}
    \xymatrix{
        \Omega^p(A) \ar[r]^\rho \ar[d]^{d} & 
        \Gamma^{(p)} \ar[d]^\Delta&\\
        \Omega^{p+1}(A) \ar[r]^\rho & 
        \Gamma^{(p+1)}&
    }
    \end{displaymath}
\end{theorem}

\section*{Non-commutative Geometry}
\subsection{Spectral Triples}
Non-commutative geometry is an attempt to generalise theorems of differential geometry
to non-commutative algebras. The basic object of study in non-commutative differential
geometry is the \emph{spectral triple}. A spectral triple plays the role of a 
``manifold". The definition is as follows.
\begin{definition}
    A spectral triple is a triple, $(\A,\Hilb,\D)$. Where,
    $\A$ is a $*$-algebra of bounded operators on a Hilbert space $\Hilb$, and
    $\D$ is a densely defined unbounded self adjoint operator on $\Hilb$, such
    that $[\D,a] \in \mathcal{B}(\Hilb)$ for all $a \in \A$ and $(\D-\lambda)^{-1}$
    is compact for all $\lambda \in \Cplx\setminus \Rl$.
\end{definition}

This is supposed to be an abstract version of a spin manifold. The commutative case
is when $M$ is a Riemannian manifold with a spin bundle $S$, then we have $\A = C^\infty(M)$,
acting by pointwise multiplication on the bundle $\Hilb = L^2(S)$, and $\D$
is the dirac operator.

It can be shown, under fairly general conditions, that when $\A$ is commutative, then 
such a spin manifold can be constructed. Hence a spectral triple deserves
the name of ``non-commutative manifold".
\subsection{Connes Differentials}
Given a spectral triple, $(\A,\Hilb,\D)$, we would like to construct an ``exterior algebra"
on $\A$. Connes does this by identifying the $1$-form $da$ with $[D,a]$.

Since $[D,a]$ is a derivation on $\A$, by the universal property we have a map, $\pi:\Omega\A\rightarrow \mathcal{B}(\Hilb)$
given by $\pi(ada_1da_2\cdots da_n) = a[D,a_1][D,a_2]\cdots[D,a_n]$.

One may then na\"ively define the algebra of differential forms as $\pi(\Omega \A)$,
but this does not work since there exists $a \in \Omega \A$ such that $\pi(a) = 0$
but $\pi(da) \neq 0$. These are called ``junk forms" and we must factor them out to get
a good differential algebra. Hence, define
\begin{theorem}
    Let $J_0$ be the graded ideal of $\Omega \A$ defined by 
    \begin{equation*}
        J_0^{(p)} = \{a \in \Omega^p(\A) \;:\; \pi(a) = 0\}
    \end{equation*}
    And define $J^{(p)} = J_0^{(p)} + dJ_0^{(p)}$. Then $J = \bigoplus_p J^{(p)}$.
\end{theorem}

Now we can define the algebra of Connes' forms,
\begin{equation*}
    \Omega_{\D}\A = \frac{\Omega \A}{J} \cong \frac{\pi(\Omega \A)}{\pi(dJ_0)}
\end{equation*}

$\Omega_\D \A$ is naturally graded by the gradings on $\Omega \A$ and $J$, with the 
space of $p$-forms being $\Omega^p_\D \A = \Omega^p(\A)/J^{(p)}$.

Since $J$ is a differential ideal, the operator $d$ on $\Omega \A$
extends to $\Omega_\D \A$. 

\end{document}
