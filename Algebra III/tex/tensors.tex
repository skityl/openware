% Copyright © 2015 Edward McDonald. All Rights Reserved.
% !Tex root = Algebra III.tex

\newcommand{\Cliff}{\operatorname{Cliff}}
\newcommand{\Cl}{C\ell}
%\newcommand{\im}{\operatorname{im}}
\newcommand{\com}{\mathrm{com}}
\newcommand{\Hom}{\operatorname{Hom}}
\newcommand{\A}{\mathcal{A}}
\newcommand{\Hilb}{\mathcal{H}}
\newcommand{\D}{\mathcal{D}}
\newcommand{\End}{\operatorname{End}}
\newcommand{\Mult}{\operatorname{Mult}}
\newcommand{\isom}{\cong}

\setlength\parindent{0pt}

%
\section{Tensors} % (fold)
\label{sec:tensors}

%%
\subsection{Introduction}
Tensors are a vital tool in modern mathematics. They see
frequent use in representation theory, module theory, linear algebra, differential
geometry and physics.

Unfortunately for a beginner, it can be very difficult to start learning
the theory of tensors. There are several reasons for this.
Most importantly is that there is a bewildering variety of 
definitions of tensors in the literature. It is not obvious
how these definitions relate to each other or which definition is ``best".

Another issue is that it is very hard to find a clear exposition of even one 
of these definitions. For example, the oldest definition, which is still
widely used in physics books, goes something like:
\begin{displayquote}
A tensor is a quantity that transforms according to the tensor 
transformation law.
\end{displayquote}
These books will then give a thoroughly unsatisfactory explanation of this
``tensor transformation law". There are many questions left unanswered.
What is a ``quantity"? What does it mean to ``transform" anyway?

Physics students often do not mind this 
vagueness, since it is clear from context which quantities are thought of as tensors,
and the transformation law can be learned from examples so it does not need a
precise statement.

This can be infuriating to a mathematics student who is accustomed to
precise definitions. However, if one opens a mathematics textbook to find a definition
of a tensor, one is faced with many pages of esoteric abstract algebra. Algebraists
have abstracted the notion of a tensor so far that it would be unrecognisable
to a physicist, and learning the modern mathematical definition of a tensor
can be difficult if one does not understand the historical notion which
motivates it.

The purpose of these notes is to give a clear exposition of the concept of a tensor.
The general philosophy is that even though there are many different definitions,
none is ``best". In fact, in order to get a ``stereoscopic" view
of the subject, all different definitions must be learned and understood, as
well as the relationships between the different definitions.

One thing that I hope to make clear is that, if one is interested
in studying multilinear maps, \emph{tensors are inevitable}. What I mean by
this is that one cannot study multilinear algebra without at least implicitly
defining a tensor.

\subsection{An initial definition}

Multilinear maps are ubiquitous in mathematics, and we shall take for granted 
that this makes them worthy of study. Our first definition of tensors
is very clear: a tensor is simply a multilinear map.

Let $k$ be a field, and let $V$ be a finite dimensional vector space over $k$.
We define the \emph{tensor space} as follows,
\begin{definition}
    Let $r,s \geq 0$ be non-negative integers. Define
    \begin{equation*}
        T^{r,s}(V) = \{f:\underbrace{V\times \cdots \times V}_{r\text{ times}} \times \underbrace{V^* \times \cdots \times V^*}_{s\text{ times}} \rightarrow k\;:\; f\text{ is multilinear}\}.
    \end{equation*}
    The elements of $T^{r,s}(V)$ are called tensors of type $(r,s)$.
    It is easily verified that $T^{r,s}(V)$ forms a vector space over $k$, with
    addition of tensors given by pointwise addition of multilinear functions.    
\end{definition}

We denote by $\End_k(V)$ the space of $k$-linear maps from $V$ to $V$. Our
first theorem for this notes relates linear maps to multilinear maps.
\begin{proposition}
    There is a natural and canonical isomorphism,
    \begin{equation*}
        \End_k(V) \cong T^{1,1}(V).
    \end{equation*}
\end{proposition}
\begin{proof}
    The idea here is that to specify a linear map $T:V\rightarrow V$
    you need to give $T(v) \in V$ for each $V$, and it is enough
    to know the value of $w(T(v))$ for each $w \in V^*$.

    We build an isomorphism $\psi:\End_k(V)\rightarrow T^{1,1}(V)$ as follows. 
    For $T \in \End_k(V)$, $v \in V$ and $w \in V^*$, define
    \begin{equation*}
        \psi(T)(v,w) = w(T(v)).
    \end{equation*}
    It is clear that $\psi(T) \in T^{1,1}(V)$. We now need to show that $\psi$ is
    an isomorphism. To do this, we define an inverse map $\varphi$. 

    For $S \in T^{1,1}(V)$, and $v \in V$, define $\varphi(S)(v)$
    as the unique vector such that $w(\varphi(S))(v) = S(v,w)$ for
    all $w \in V^*$. The existence of this vector is guaranteed by 
    the fact that $V$ is finite dimensional.

    It is an exercise to check that $\psi\circ \varphi = \mathrm{id}_{T^{(1,1)}(V)}$
    and $\varphi \circ \psi = \mathrm{id}_{\End_k(V)}$.
\end{proof}

We now define the \emph{tensor product}, which will allow
us to provide a simple description of every tensor space. 
\begin{definition}
    Let $S \in T^{r,s}(V)$ and $T \in T^{p,q}(V)$. Define $S \otimes T \in T^{r+p,s+q}(V)$
    as
    \begin{equation*}
        (S\otimes T)(v_1,\ldots,v_{r+p},w_1,\ldots,v_{s+q}) = S(v_1,\ldots,v_r,w_1,\ldots,w_s)T(v_{r+1},\ldots,v_{r+p},w_{s+1},\ldots,w_{s+q}).
    \end{equation*}
    for $v_i \in V$, $i = 1,\ldots,r+p$ and $w_j\in V^*$, $j = 1,\ldots,s+q$.
\end{definition}

I would like to make a point about canonical identifications. In what follows,
we will identify a finite dimensional vector space $V$ with its double dual $V^{**}$
in a standard way: the vector $v$ will be associated with the linear map consisting
of evaluation at $v$. I wish to argue that this map is \emph{natural}, where I mean 
the word natural in a specific technical sense. Namely, for any two vector
spaces $V$ and $W$, and a linear map $T:V\rightarrow W$, we have a corresponding
map $T^{**}:V^{**}\rightarrow W^{**}$, and if $\varphi$ denotes the isomorphism from
$V$ to $V^{**}$ and $\psi$ is the isomorphism from $W$ to $W^{**}$, then $T^{**} = \psi \circ T \circ \varphi$.

This may all seem like abstract nonsense: but the idea is that we have a \emph{natural transformation}
connecting $V$ and $V^{**}$, which means that we can identify them freely.

With this in mind, we note that
\begin{align*}
    V^* &\cong T^{1,0}(V)\\
    V   &\cong T^{0,1}(V).
\end{align*}

Since $V$ is finite dimensional, we can choose a basis $\{e_1,\ldots,e_n\}$,
where $n$ is the dimension of $V$. We can choose a basis for $V^*$ which is compatible
with this basis in a very logical way. Define the linear functionals $e^{i}$, $i = 1,\ldots,n$
as follows,
\begin{equation*}
    e^{i}(e_j) = \begin{cases}
        1\text{ if }i = j\\
        0\text{ otherwise.}
    \end{cases}
\end{equation*}
The set $\{e^1,e^2,\ldots,e^n\}$ spans $V^*$ and is called the dual basis to $\{e_1,e_2,\ldots,e_n\}$.

Using tensor products and these bases, we can find a basis for every tensor space.

\begin{proposition}
    A basis for $T^{r,s}(V)$ is given by
    \begin{equation*}
        \{ e^{i_1}\otimes e^{i_2}\otimes\cdots\otimes e^{i_r} \otimes e_{j_1}\otimes e_{j_2} \otimes \cdots\otimes e_{j_s}\;:\; 1\leq  i_1,\ldots,i_r,j_1,\ldots,j_s \leq n\}
    \end{equation*}
\end{proposition}

%%
\subsection{Tensor products of vector spaces}
The content and definition of the previous section allows us to study
multilinear maps on a single vector space $V$, however
often we are interested in multilinear maps between multiple vector spaces. How 
do we use this definition to study maps from $V\times U \rightarrow W$, where $V$,$U$
and $W$ are all vector spaces over $k$, of potentially different dimension?

We could in principle define tensor spaces, $T(U,V,W^*)$ of multilinear maps
from $U\times V\times W^*$ to $k$, but this point of view quickly becomes cumbersome,
and there is a better way.

Note that in the previous section we could conveniently describe any tensor
space by using the simple machinery of a tensor product. We can now
use a similar idea, in greater abstraction, to build all spaces of multilinear maps
between arbitrary numbers of different vector spaces. 

The tensor product of two vector spaces is determined uniquely by a \emph{universal 
property}. However, this is very abstract and difficult to approach at
first. Therefore, we will construct the tensor product explicitly,
in order to explain the universal property.
\begin{definition}
    Let $V$ and $W$ be two finite dimensional vector spaces over a
    field $k$. We define the \emph{tensor product} $V \otimes W$
    to be the vector space spanned by all the symbols of the form
    \begin{equation*}
        \{ v\otimes w\;:\; v \in V, w \in W\}.
    \end{equation*}
    Subject to the restriction that,
    \begin{align*}
        \alpha(v\otimes w) = (\alpha v)\otimes w &= v\otimes (\alpha w), \text{ for all }\alpha \in k, v \in V, W\in W\\
        (u+v)\otimes w &= u\otimes w+v\otimes w, \text{ for all }u,v \in V, w \in W\\
        v \otimes (w+x) &= u\otimes w + u\otimes x, \text{ for all }v \in V, w,x \in W.
    \end{align*}
\end{definition}
Note that, in the previous section, the tensor product $v\otimes w$
of two vectors has a specific meaning: it was a multilinear map on $V^*\times V^*$.
But at the moment, we are taking $v\otimes w$ to be simply a formal symbol. Later
on we will see how to recover a multilinear map. Elements of $V\otimes W$
of the form $v\otimes w$ are called \emph{elementary tensors}. An arbitrary element
of $V\otimes W$ is a sum of elementary tensors, $v_1\otimes w_1+v_2\otimes w_2 + \cdots + v_k\otimes w_k$.
It is a common mistake to think that $V\otimes W$ consists only of elementary tensors.

We now cover the universal property. Universal properties are beloved
by modern algebraists, because they provide an elegant way of defining an object.
However, to a beginner, they appear abstruse or even unnecessary. A universal
property is used to define an object that is the ``largest" or ``smallest"
of a given type. 

The universal property for tensor products can be taken to mean two things:
\begin{itemize}
    \item{} The tensor product $V\otimes W$ is the largest vector space
    which is the image of a multilinear map from $V\times W$.
    \item{} Any multilinear map from $V\times W$ can be considered
    as a linear map from $V\otimes W$.
\end{itemize}
The beauty of the universal property is that it explains how these two points
of view are equivalent. 

So finally we state the universal property:
\begin{theorem}
    Suppose $U$,$V$ and $W$ are finite dimensional vector spaces over $k$.

    Note that there is a multilinear ``inclusion" map, $\iota:U\times V\rightarrow U\otimes V$,
    where $\iota(u,v) = u\otimes v$.

    If $\theta:U\times V\rightarrow W$ is a multilinear map, there exists
    a unique linear map $\tilde{\theta}:U\otimes V\rightarrow W$ such that the following
    diagram commutes,
    \begin{displaymath}
    \xymatrix{
        {U\times V} \ar[r]^\iota \ar[rd]^\theta &
         {U\otimes V} \ar@{.>}[d]^{\tilde{\theta}}&\\
         &
         W
    }    
    \end{displaymath}
    
\end{theorem}
Another way to see the universal property is as an isomorphism of $k$ spaces,
namely
\begin{equation*}
    \Mult(U\times V,W) \isom L(U\otimes V,W)
\end{equation*}
Where the left hand side is multilinear maps from $U\times V$ to $W$, and
the right hand side is linear maps from $U\otimes V$ to $W$.

The next proposition may seem innocuous, but it is crucial to connecting
this definition of a tensor with the definition in the previous section.
This is also practice at using the universal property.
\begin{proposition}
    There is a natural isomorphism,
    \begin{equation*}
        (V\otimes W)^* \cong V^*\otimes W^*.
    \end{equation*}
\end{proposition}
\begin{proof}
    By the universal property, any linear map on $U\otimes V$
    is determined by its values on elementary tensors. So let $v\otimes w \in V\otimes W$,
    for $f\otimes g \in V^*\otimes W^*$, define $(f\otimes g)(v\otimes w) = f(v)g(w)$,
    and extend this to all of $V^*\otimes W^*$ by linearity. 
\end{proof}

By the universal property,
\begin{equation*}
    T^{1,1}(V) = \Mult(V\times V^*,k) \isom L(V\otimes V^*,k) = (V\otimes V^*)^* \isom V^*\times V. 
\end{equation*}
And this shows how section 2 is a special case of section 3.

Similarly, we may obtain an isomorphism,
\begin{proposition}
    $L(V,W) \isom V^* \otimes W$
\end{proposition}
\begin{proof}
    We construct the map $\psi: V^* \otimes W \rightarrow L(V,W)$. By the universal
    property, this map is determined by its values on elementary tensors,
    so we only need to consider a multilinear map $V^*\times W \rightarrow L(V,W)$.
    Given $f \in V^*$ and $w \in W$, define $\psi(f\otimes w)$ as the map
    $v\mapsto f(v)w$.
\end{proof} 

It is then straightforward to define higher tensor products: such as $U \otimes V \otimes W$.

%%
\subsection{Tensor Algebras}
The Tensor Algebra is a phenomenally useful tool, and once we define it, it
will be used to construct a variety of other algebras.

Let $V$ be a finite dimensional vector space over a field $k$. The
\emph{tensor algebra} of $V$ is the ``largest algebra generated by $V$".
Accordingly, it is defined by a universal property. However, first we
give an explicit construction.

We must define what it meant by algebra. An algebra $A$ over a field
$k$ is a ring that is also a $k$ vector space. The ring multiplication
must be compatible with the scalar multiplication in the following sense: 
\begin{align*}
    \alpha(vu) = (\alpha v)u = v(\alpha u)
\end{align*}
for all $v,u \in A$ and $\alpha \in k$. 

An algebra homomorphism is a ring homomorphism that is also $k$-linear.

An explicit construction of the tensor algebra follows,
\begin{definition}
    For $n > 0$, define
    \begin{equation*}
        V^{\otimes n} := \underbrace{V \otimes \ldots \otimes V}_{n\text{ times}}
    \end{equation*}
    and $V^{\otimes 0} := k$. The \emph{tensor algebra}, $T(V)$, is defined as 
    the direct sum of vector spaces,
    \begin{equation*}
        T(V) := \bigoplus_{n=0}^\infty V^{\otimes n}.
    \end{equation*}
    This is a priori a vector space over $k$. However it is also an algebra, an
    arbitrary element of $T(V)$ is of the form,
    \begin{equation*}
        v_0+v_1+v_2+\cdots+v_m
    \end{equation*}
    for $v_i \in V^{\otimes i}$, $i = 1,\ldots m$. However, we can multiply 
    elements like this formally. Suppose $v \in V^{\otimes i}$, and $u \in V^{\otimes j}$
    are elementary tensors, then $v \otimes u$ is an elementary tensor in $V^{\otimes i+j}$.

    Then the multiplication is extended to all of $T(V)$ by linearity.
    Note that the identity in $T(V)$ is simply the identity in $k$.

    Frequently the $\otimes$ symbol is omitted, so we write $uv = u \otimes v$.
\end{definition}

One may think of $T(V)$ as being the algebra built up from
formal multiplications of elements of $V$, with the assumption
that multiplication is $k$-linear and associative. We place no other assumptions on $T(V)$,
so $T(V)$ is somehow the ``largest" algebra built from $V$.

The universal property of tensor algebras says that:
\begin{itemize}
    \item{} $T(V)$ is the largest algebra that is the image of $V$ as a linear map.
    \item{} Any linear map from $V$ to an algebra $A$, can be extended to an algebra homomorphism from $T(V)$ to $A$.
\end{itemize}

More precisely,
\begin{proposition}
    Note that since $V = V^{\otimes 1}$, there is a natural
    inclusion map $\iota: V \rightarrow T(V)$.

    Suppose that $A$ is a $k$-algebra, and let $\theta:V\rightarrow A$
    be any $k$-linear map. Then there exists a unique algebra homomorphism
    $\tilde{\theta}$ such that the following diagram commutes,
    \begin{displaymath}
    \xymatrix{
        V \ar[r]^\iota \ar[rd]^\theta &
        T(V) \ar@{.>}[d]^{\tilde{\theta}} &\\
        &
        A
    }
    \end{displaymath}
\end{proposition}

%%
\subsection{Algebras Built from the Tensor Algebra}
%%%
\subsubsection{Quotients of algebras}

An ideal $I \unlhd A$ is an (two-sided) ideal of $A$ as a ring, and a $k$-subspace of $A$. 

Given an ideal $I \unlhd A$, we can describe the quotient algebra, $A/I$,
as being the quotient ring $A/I$, where the $k$ vector space structure
is given by $\alpha(x+I) = (\alpha x) + I$ for $x + I \in A/I$
and $\alpha \in k$. This is well defined since $I$ is a $k$-subspace of $A$.

There is a projection map $\pi:A\rightarrow A/I$, that maps $x \mapsto x+I$.

The feature of quotient spaces that will be relevant to us is the universal property.
This is in fact an elegant way of defining the quotient space. In brief, what it
means is that $A/I$ is the largest algebra which is the image of an algebra homomorphism
from $A$ with kernel containing $I$.
\begin{proposition}
    Suppose that $A$ is an algebra over $k$, and $I \unlhd A$ is an ideal.

    Let $M$ be another $k$ algebra, and $\varphi:A\rightarrow M$
    is an algebra homomorphism such that $I \subseteq \ker\varphi$.
    Then there is a unique map $\tilde{\varphi}$ such that the following
    diagram commutes,
    \begin{displaymath}
    \xymatrix{
        A \ar[r]^\pi \ar[rd]^\varphi&
        A/I \ar@{.>}[d]^{\tilde{\varphi}}&\\
        &
        M
    }
    \end{displaymath}
\end{proposition}

Now we will use this property extensively. 

%%%
\subsubsection{The exterior algebra}

$T(V)$ was built from all formal multiplications of elements of $V$. We 
can place additional restrictions on this multiplication. In the \emph{exterior algebra}
construction, we require that multiplication is \emph{alternating}.
\begin{definition}
    Let $V$ be a finite dimensional vector space over $k$. 
    Let $I$ be the ideal of $T(V)$ generated by all elements of the form
    $v \otimes v$, for $v \in T(V)$.
    Define,
    \begin{equation*}
        \bigwedge V := \frac{T(V)}{I}.
    \end{equation*}
    This is called the exterior algebra of $V$, and we denote
    the image of $u \otimes v$ in $\bigwedge T$ as $u \wedge v$, that is
    \begin{equation*}
        u \wedge v := u \otimes v + I.
    \end{equation*}

    See that by construction, $u \wedge u = 0$, and since $(u+v)\wedge(u+v) = 0$,
    we conclude that $u\wedge v = - v \wedge u$. An algebra where the product
    is antisymmetric like this is called alternating.
\end{definition}

By the universal property for quotients, $\bigwedge V$ satisfies the following
universal property.
\begin{proposition}
    See that we still have an inclusion map, $\iota:V\rightarrow \bigwedge V$.

    Let $A$ be any algebra, and let $\varphi:V\rightarrow A$
    be a $k$-linear map such that $\varphi(v)\varphi(v) = 0$. Then there exists
    a unique algebra homomorphism $\tilde{\varphi}$ such that the following diagram
    commutes,
    \begin{displaymath}
    \xymatrix{
        V \ar[r]^\iota \ar[rd]^\varphi&
        \bigwedge V \ar@{.>}[d]^{\tilde{\varphi}}&\\
        &
        A
    }
    \end{displaymath}
\end{proposition}

One may freely think of $\bigwedge V$ as the algebra
generated by symbols of the form $v_1 \wedge v_2 \wedge \cdots \wedge v_m$,
where we say that the $\wedge$ operation is bilinear and associative, and $v \wedge v = 0$
for any $v \in \wedge V$.

%%%
\subsubsection{The symmetric algebra}
In a very similar way, we may define the \emph{symmetric tensor algebra}.

$T(V)$ was built from all formal multiplications of elements of $V$. We 
can place additional restrictions on this multiplication. In the \emph{symmetric algebra}
construction, we require that multiplication is \emph{commutative}.
\begin{definition}
    Let $V$ be a finite dimensional vector space over $k$. 
    Let $J$ be the ideal of $T(V)$ generated by all elements of the form
    $v \otimes u - u\otimes v$, for $u,v \in T(V)$.
    Define,
    \begin{equation*}
        S(V) := \frac{T(V)}{J}.
    \end{equation*}
    This is called the symmetric tensor algebra of $V$, and we denote
    the image of $u \otimes v$ in $S(T)$ as $uv$, that is
    \begin{equation*}
        uv := u \otimes v + J.
    \end{equation*}

    See that by construction, $uv = vu$.
\end{definition}

By the universal property for quotients, $S(V)$ satisfies the following
universal property.
\begin{proposition}
    See that we still have an inclusion map, $\iota:V\rightarrow S(V)$.

    Let $A$ be any algebra, and let $\varphi:V\rightarrow A$
    be a $k$-linear map such that $\varphi(u)\varphi(v) = \varphi(v)\varphi(u)$. Then there exists
    a unique algebra homomorphism $\tilde{\varphi}$ such that the following diagram
    commutes,
    \begin{displaymath}
    \xymatrix{
        V \ar[r]^\iota \ar[rd]^\varphi&
        S(V) \ar@{.>}[d]^{\tilde{\varphi}}&\\
        &
        A
    }
    \end{displaymath}
\end{proposition}

%%%
\subsubsection{Clifford algebras}
The constructions of the exterior algebra and the symmetric algebra are nearly
identical. In fact, to write the section on the symmetric algebra I simply
copy-pasted the construction of the exterior algebra, and changed a few words.
The purpose of this is to show the great utility of the tensor algebra: any
algebra built from a vector space $V$ can be constructed as a quotient
of $T(V)$. We will use exactly this trick to construct clifford algebras. 

However before we jump into the definition
I should say something about the historical motivation behind this concept.

The laplacian operator on $\Rl^2$ is given by
\begin{equation*}
    \Delta = -\frac{\partial^2}{\partial x^2}-\frac{\partial^2}{\partial y^2}.
\end{equation*}

For reasons to do with quantum mechanics, the famous physicist P. Dirac sought a linear
operator $D$ such that
\begin{equation*}
    D^2 = \Delta.
\end{equation*}
Dirac wanted $D$ to be a first order partial differential operator, that is,
a linear combination,
\begin{equation*}
    D = \gamma_1 \frac{\partial}{\partial x}+\gamma_2 \frac{\partial}{\partial y}
\end{equation*}
for some coefficients $\gamma_1$ and $\gamma_2$. However, if one attempts to find
such an operator one ends up with the system of equations:
\begin{align*}
    \gamma_1^2 = \gamma_2^2 &= -1\\
    \gamma_1\gamma_2+\gamma_2\gamma_1 &= 0.
\end{align*}
But this system has no solution over the complex numbers, because
complex numbers commute. 

Dirac had the brilliant idea that while these equations do not have complex solutions,
they do have matrix solutions.
If we choose
\begin{align*}
    \gamma_1 &= \begin{pmatrix} 0 & -1 \\ 1 & 0 \end{pmatrix}\\
    \gamma_2 &= \begin{pmatrix} 0 & i \\ i & 0 \end{pmatrix}
\end{align*}
then the problem is solved. 

However, it would be nice to be able to do this in an arbitrary number
of dimensions, rather than just $2$.

But first,
we must say something about quadratic forms.
\begin{definition}
    A symmetric bilinear form $b:V\times V\rightarrow k$
    is a map that is linear in both arguments (In other words, $b \in T^{2,0}(V)$
    or $b \in V^* \otimes V^*$) and $b(u,v) = b(v,u)$ for all $u,v \in V$. 

    Associated to any bilinear form is a \emph{quadratic form},
    \begin{equation*}
        Q(v) = b(v,v).
    \end{equation*}
\end{definition}


Basically, the clifford algebra $\Cl(V,b)$
on a vector space $V$ and a symmetric bilinear form $b \in V^* \otimes V^*$
is going to be the algebra generated by $V$ modulo the relation
that $uv+vu = -2b(u,v)$, or equivalently that $u^2 = -b(u,u)$. 
\begin{definition}
    Let $V$ be a finite dimensional vector space over $k$, and 
    let $b \in T^{2,0}(V) = V^* \otimes V^*$ be a symmetric bilinear form.

    Let $K$ be the ideal of $T(V)$ generated by all elements
    of the form $u\otimes u + b(u,u)1$, where $1$ is the identity
    in $T(V)$. Then, the \emph{clifford algebra} associated to
    $V$ and $b$ is defined as
    \begin{equation*}
        \Cl(V,b) := \frac{T(V)}{K}
    \end{equation*}
    We denote the image of $u \otimes v$ in $\Cl(V,b)$ as $uv$.
\end{definition}


%%
\subsection{Tensor products of vector bundles}

So far we have been working in single vector spaces
over an arbitrary field $k$. A \emph{vector bundle} is a powerful
generalisation of a vector space, which consists of a collection
of vector spaces indexed by a topological space. 
\begin{definition}
    A vector bundle is a triple $(E,\pi,X)$ where $E$
    and $X$ are topological spaces, and $\pi:E\rightarrow X$
    is a continuous surjection.
\end{definition}

%%
\subsection{Tensor calculus}
...

%%
\subsection{Tensor products of topological vector spaces}

All of the vector spaces that we have defined up to this point
have been finite dimensional. There is a good reason for this:
infinite dimensional spaces require a more advanced theory.

We are interested in three classes of infinite dimensional
spaces, which we classify by their topology. In order from
least to most general,
\begin{enumerate}
\item{} Hilbert spaces have a topology induced by an inner product.
\item{} Banach spaces have a topology induced by a norm.
\item{} Locally convex spaces have a topology induced by a family of semi-norms.
\end{enumerate}
finish writing this later...

%%
\subsection{Tensor products of modules}
..

%%
\subsection{Monoidal categories}
..
