\documentclass{owmaths}

\usepackage{owshortcuts}
\usepackage[all]{xy}
\usepackage{csquotes}

\begin{document}

\subject{Misc.}
\author{Edward McDonald}
\title{Filters and Applications}
\studentno{616}

\newcommand{\Cliff}{\operatorname{Cliff}}
\newcommand{\Cl}{C\ell}
\newcommand{\im}{\operatorname{im}}
\newcommand{\com}{\mathrm{com}}
\newcommand{\Hom}{\operatorname{Hom}}
\newcommand{\A}{\mathcal{A}}
\newcommand{\Hilb}{\mathcal{H}}
\newcommand{\D}{\mathcal{D}}
\newcommand{\End}{\operatorname{End}}
\newcommand{\Mult}{\operatorname{Mult}}
\newcommand{\isom}{\cong}

\setlength\parindent{0pt}

\section{Introduction}    
A \emph{filter} is a mathematical concept from order theory that is
very useful in topology and mathematical logic. As is well known, to study
continuity and convergence on a topological space
it is not enough to study convergence of sequences. Instead, the concept
of a \emph{generalised sequence} also known as a \emph{net} is introduced. Nets are 
very useful, and their similarity to sequences makes them very easy to understand.

However, nets are somewhat unnatural to use as the concept of a subnet is unwieldy. 
An alternative is provided by the concept of a filter. Filters are in a sense more natural 
to use than nets for topology.

Filters also naturally arise in logic. The most spectacular
application of filters to metamathematics is probably in the \emph{ultrapower
construction}. This is an exceptionally useful tool and is used
to construct the hyperreal numbers, a powerful and fascinating number system
which contains the real numbers and a plentiful supply of infinitesimals.

\section{Definitions}
\subsection{Filters}
Here we define a filter in the greatest possible abstraction. This will
actually be more abstract than is needed for these notes but certain features
of the definition are made clear from this point of view.

\begin{definition}
    Let $(P,\leq)$ be a partially ordered set. A \emph{filter} is a 
    proper subset $\mathcal{F} \subset P$
    satisfying the following two properties:
    \begin{enumerate}
        \item{} If $x \in \mathcal{F}$ and $x \leq y$, then $y \in \mathcal{F}$.
        \item{} If $x,y \in \mathcal{F}$, then there exists $z \in \mathcal{F}$
        with $z \leq x$ and $z \leq y$.
    \end{enumerate}
    The first property is called \emph{upward closure}. A set with property
    $1$ is called upward closed. 
    
    The second property is called \emph{downward direction}. A set
    with property $2$ is called downward directed.
\end{definition}
Note that we have defined a filter as a proper subset. That is, we
cannot have $\mathcal{F} = P$. This is not followed by all authors. Some
authors say that a filter can be any subset, and a filter which is a proper
subset is called a proper filter. All filters in these 
notes are proper filters.

A convenient way of specifying a filter is by a filter base:
\begin{definition}
    Let $(P,\leq)$ be a partially ordered set. A subset of $P$
    which is downward directed is called a \emph{filter base}.
    
    That is, a subset $A \subset P$ is a filter base if for any $x,y \in A$, 
    there exists a $z \in A$ with $z \leq x$ and $z \leq y$.
    
    Obviously, a filter base which is upward closed is a filter. We can generate
    a filter from a filter base by taking the upward closure,
    \begin{equation*}
        \mathcal{F}_A := \{ x \in P\;:\;\exists z \in A, z \leq x\}
    \end{equation*}
    is called the upward closure of $A$. If $A$ contains
    no minimal element of $P$, then it is easy to check that $\mathcal{F}_A$
    is a filter.
\end{definition}

It is very easy to find examples of filters.
\begin{example}
    Let $(P,\leq)$ be a partially ordered set. Let $p \in P$ be such
    that $p$ is not a minimal element of $P$. Define
    \begin{equation*}
        \mathcal{F} := \{x \in P\;:\; p\leq x\}.
    \end{equation*}    
    Then $\mathcal{F}$ is a filter for $(P,\leq)$.
\end{example}
\begin{example}
    Let $X$ be a set, and let $\mathcal{P}(X)$ be the power set
    of $X$, partially ordered by inclusion. If $x \in X$, then
    \begin{equation*}
        \mathcal{F} := \{ A \in \mathcal{P}(X)\;:\; x \in A\}
    \end{equation*}
    is a filter for $(\mathcal{P}(X),\subseteq)$. This is called a principal
    filter for $x$.\\
    
\end{example}
\begin{example}
    Let $X$ be any infinite set, Let 
    \begin{equation*}
        \mathcal{F} := \{ A \in \mathcal{P}(X)\;:\;|X\setminus A| < \infty\}.
    \end{equation*}
    Then $\mathcal{F}$ is a filter for $(\mathcal{P}(X),\subseteq)$. This
    is called the cofinite filter on $X$.
\end{example}

The previous two examples were examples of filters on the partially ordered
set $\mathcal{P}(X)$ for some set $X$. The filters we are mostly
interested in for these notes are of this form, so we single
them out and give them a name:
\begin{definition}
    A \emph{filtered set} $(X,\mathcal{F})$ is a set $X$
    with a filter $\mathcal{F}$ for $(\mathcal{P}(X),\subseteq)$. 
    We say that $\mathcal{F}$ is a filter of $X$.
\end{definition}

We give a very easy proposition which characterises filtered sets.
\begin{proposition}
    A subset $\mathcal{F} \subseteq \mathcal{P}(X)$ is a filter of $X$
    if and only if it satisfies the following three properties:
    \begin{enumerate}
    \item{} $\emptyset \notin \mathcal{F}$
    \item{} $\mathcal{F}$ is closed under intersection. That is, if $A,B\in \mathcal{F}$,
    then $A \cap B \in \mathcal{F}$.
    \item{} $\mathcal{F}$ is upward closed. That is, if $A \in \mathcal{F}$
    and $A \subseteq B$ then $B \in \mathcal{F}$.
    \end{enumerate}
\end{proposition}

\subsection{Ultrafilters}
An ultra filter is a filter that is not contained in any larger
filter, that is,
\begin{definition}
    An ultrafilter on $(P,\leq)$ is a filter $\mathcal{U}$
    such that if $\mathcal{F}$ is a filter on $(P,\leq)$, 
    with $\mathcal{U} \subseteq \mathcal{F}$, then $\mathcal{U} = \mathcal{F}$.
\end{definition}

There are simple examples of ultrafilters. For example, if $X$
is a set with $x \in X$ and $\mathcal{F}$ is the subset of $\mathcal{P}(X)$
consisting of sets containing $x$. Then $\mathcal{F}$ is an ultrafilter.
An ultrafilter built as the set of all sets containing some point is called
a principal ultrafilter.

It is not at all obvious that non-principal ultrafilters should exist. However,
they exist in great abundance. The following theorem
shows that this is a consequence of Zorn's lemma:
\begin{theorem}
\label{ultrafilter}
    Let $(P,\leq)$ be a partially ordered set with a minimal element $0$. Let
    $\mathcal{F}$ be a filter on $(P,\leq)$. Then there is an ultrafilter 
    $\mathcal{U}$ on $(P,\leq)$ containing $\mathcal{F}$.
\end{theorem}
\begin{proof}
    Let $V$ be the collection of filters on $(P,\leq)$ containing $\mathcal{F}$.
    We will use Zorn's lemma to show that $V$ has a maximal element, which must
    be an ultrafilter.
    
    Clearly $\mathcal{F} \in V$, so $V$ is not empty. Now let $S \subseteq V$
    be a linearly ordered subset. Let $K = \cup S$. 
    
    If $x,y \in K$, then $x,y \in A$ for some $A \in S$. But
    $A$ is a filter, so there exists some $z \in A$ with $z \leq x$ and $z \leq y$,
    hence $z \in K$.
    
    Also, if $x \leq w$, then $w \in A$, hence $w \in K$.
    
    Since each $A \in S$ is proper subset, we cannot have $0 \in A$ for any $A \in S$.
    Hence, $0 \notin K$. So $K \neq P$.
    
    Thus, $K$ is a filter containing $\mathcal{F}$. 
    
    Hence any linearly ordered subset of $V$ has an upper bound. So by Zorn's lemma
    $V$ has a maximal element. Let $\mathcal{U}$
    be this maximal element. $\mathcal{U}$ is an ultrafilter by construction,
    and if $\mathcal{U}$ is properly contained in any other filter, then it would not
    be maximal for $V$. Hence $\mathcal{U}$ is an ultrafilter.
\end{proof}
\begin{example}
    Let $X$ be an infinite set, and let $\mathcal{F}$ be the filter
    of cofinite subsets of $X$. By theorem \ref{ultrafilter},
    $\mathcal{F}$ is contained in an ultrafilter $\mathcal{U}$.
    
    By construction, this ultrafilter is not principal.
\end{example}

There is a useful characterisation of ultrafilters among filters on sets, described
by the next proposition,
\begin{proposition}
    Let $\mathcal{F}$ be a filter of a set $X$. Then $\mathcal{F}$
    is an ultrafilter if and only if it has the property that if $A \subseteq X$,
    then either $A \in \mathcal{F}$ or $X \setminus A \in \mathcal{F}$.
\end{proposition}
\begin{proof}
    Suppose that $\mathcal{F}$ is an ultrafilter and there is some set $A \subseteq X$
    such that both $A \notin \mathcal{F}$ and $X\setminus A \notin \mathcal{F}$. Then
    we may form the set
    \begin{equation*}
        D := \{A \cap F\;:\;F \in \mathcal{F}\}.
    \end{equation*}
    It is easy to see that $D$ is downward directed.
    Since $X \setminus A \notin \mathcal{F}$, no subset of $X \setminus A$ is
    in $\mathcal{F}$. Hence $\emptyset \notin D$.
    
    Hence $D$
    is a filter base. The filter
    generated by $D$ must contain $A$, and hence strictly contains $\mathcal{F}$.
    But $\mathcal{F}$ is an ultrafilter, so this is a contradiction. 
    
    Conversely, suppose that $\mathcal{F}$ is a filter of $X$ that is not
    an ultrafilter. Hence there is a filter $\mathcal{U}$
    strictly containing $\mathcal{F}$. Let $A \in \mathcal{U} \setminus \mathcal{F}$. 
    Then by construction we cannot have $A \in \mathcal{F}$. However,
    if $X \setminus A \in \mathcal{F}$, then $X \setminus A \in \mathcal{U}$.
    But then $\emptyset \in \mathcal{U}$, so this is a contradiction.
\end{proof}

Hence, we can characterise ultrafilters on sets as follows,
\begin{corollary}
    A collection $\mathcal{U}$ of subsets of a set $X$ is an ultrafilter
    if and only if the following four properties hold,
    \begin{enumerate}
        \item{} $\emptyset \notin \mathcal{U}$.
        \item{} If $A,B \in \mathcal{U}$, then $A \cap B \in \mathcal{U}$
        \item{} If $A \in \mathcal{U}$, and $A \subseteq B$, then $B \in \mathcal{U}$.
        \item{} If $A \subseteq X$, then either $A \in \mathcal{U}$
        or $X \setminus A \in \mathcal{U}$.
    \end{enumerate}
\end{corollary}

\section{Applications to General Topology}
The main application of filters and ultrafilters in mathematics is to 
general topology, where they provide an excellent means of describing convergence.

Let $(X,\tau)$ be a topological space. For $x \in X$, define
$\mathcal{N}(x)$ to be the filter generated by all open sets containing $x$.

\begin{definition}
    Let $\mathcal{F}$ be a filter of the set $X$. Then we say $\mathcal{F}$
    converges to $x \in X$ if $\mathcal{N}(x) \subseteq \mathcal{F}$. We
    write this as $\mathcal{F} \rightarrow x$.
\end{definition}

\begin{proposition}
    Limits of filters are unique in a Hausdorff space.
\end{proposition}
\begin{proof}
    Suppose that $(X,\tau)$ is Hausdorff. Let $\mathcal{F}$
    be a filter on $X$ such that $\mathcal{F} \rightarrow x$
    and $\mathcal{F}\rightarrow y$ for distinct points
    $x$ and $y$. But then $\mathcal{N}(x), \mathcal{N}(y) \subseteq \mathcal{F}$.
    However, since $X$ is Hausdorff, there exist neighbourhoods separating
    $x$ and $y$. Hence there are two disjoint sets in $\mathcal{F}$,
    but this is impossible since $\mathcal{F}$ is a filter.
\end{proof} 

\begin{proposition}
    Let $A \subseteq X$. Then a point $x$ is in the closure of $A$,
    $x \in \overline{A}$, if and only if there exists
    a filter $\mathcal{F}$ on $X$ such that $A \in \mathcal{F}$
    and $\mathcal{F} \rightarrow x$. 
\end{proposition}
\begin{proof}
    Suppose that $x \in \overline{A}$. Then consider
    the set $S = \{A \cap U\;:\;U \in \mathcal{N}(x)\}$. Since
    $x \in \overline{A}$, we cannot have $\emptyset \in S$,
    and $S$ is closed under intersection since $\mathcal{N}(x)$
    is a filter. 
    
    Thus $S$ is a filterbase. Let $\mathcal{F}$ be the filter
    generated by $S$. Then $\mathcal{N}(x) \subseteq \mathcal{F}$,
    so $\mathcal{F} \rightarrow x$.
    
    Conversely, suppose that $x \in X$, and there is a filter
    $\mathcal{F} \rightarrow x$ such that $A \in \mathcal{F}$. 
    Then $\mathcal{N}(x) \subseteq \mathcal{F}$, 
    so if $U$ is a neighbourhood of $x$, then $U\cap A \neq 0$. 
    Hence $x \in \overline{A}$.
\end{proof}

\begin{definition}
    Let $X$ and $Y$ be sets, and let $\mathcal{F}$ be a filter of $X$.
    If $f:X\rightarrow Y$, the image set $\{f(F)\;:\;F \in \mathcal{F}\}$
    is a filter base, since $f(F_1\cap F_2) \subseteq f(F_1)\cap f(F_2)$. 
    Denote the filter generated by this filterbase as $f(\mathcal{F})$.
\end{definition}

\begin{proposition}
    Let $f:X\rightarrow Y$ be a function between topological spaces $X$ and $Y$.    
    $f$ is continuous if and only if, for any filter $\mathcal{F}$
    on $X$ with $\mathcal{F} \rightarrow x$, then $f(\mathcal{F}) \rightarrow f(x)$.
\end{proposition}
\begin{proof}
    First suppose that $f$ is continuous. Let $U$
    be a neighbourhood of $f(x)$. By continuity, $f^{-1}(U) \in \mathcal{F}$.
    Hence $U \in f(\mathcal{F})$. Thus $f(\mathcal{F}) \rightarrow f(x)$.
    
    Conversely, suppose that for any filter $\mathcal{F}\rightarrow x$,
    $f(\mathcal{F})\rightarrow f(x)$. 
    
    Suppose that $U$ is open in $Y$. Suppose that $x \in f^{-1}(U)$. 
    Let $\mathcal{F} = \mathcal{N}(x)$. Then $\mathcal{F}\rightarrow x$,
    hence $f(\mathcal{F})\rightarrow f(x)$. Then $U \in f(\mathcal{F})$. 
    So there is a $V \in \mathcal{N}(x)$ such that $f(V) \subseteq U$.
    Thus $x \in V \subseteq f^{-1}(U)$.
    
    This shows that $f^{-1}(U)$ is open. Hence $f$ is continuous.
\end{proof}

\begin{proposition}
    Suppose that $X$ is a set, and $\{Y_\alpha\}_{\alpha \in A}$
    is a collection of topological spaces, and $X$ is provided
    with the weak topology of the maps $f_\alpha:X\rightarrow Y_\alpha$.
    Then a filter $\mathcal{F}$ of $X$
    has $\mathcal{F}\rightarrow x$ if and only if for
    every $\alpha$, $f_\alpha(\mathcal{F})\rightarrow f_\alpha(x)$. 
\end{proposition}
\begin{proof}
    Since by definition each $f_\alpha$ is continuous, if $\mathcal{F}\rightarrow x$
    then for every $\alpha$, $f_\alpha(\mathcal{F})\rightarrow f_\alpha(x)$. 
    
    Now suppose the converse, that for every $\alpha$, 
    for some filter $\mathcal{F}$ of $X$, we have $f(\mathcal{F})\rightarrow f_\alpha(x)$
    for every $\alpha$. 
    
    The topology on $X$ is generated by sets of the form $\bigcap_{i=1}^n f_{\alpha_i}^{-1}(U_{\alpha_i})$
    for $U_{\alpha_i}$ open in $Y_{\alpha_i}$. 
    
    Let $x \in U = \bigcap_{i=1}^n f_{\alpha_i}^{-1}(U_{\alpha_i})$. Then 
    for each $i$, 
    $f_{\alpha_i}(x) \in U_{\alpha_i}$, 
    so $U_{\alpha_i} \in f_{\alpha_i}(\mathcal{F})$. Hence for each $i$
    there is some $V_i \in \mathcal{F}$
    such that $V_i \subseteq f_{\alpha_i}^{-1}(U_{\alpha_i})$.
    But since $\mathcal{F}$ is a filter,
    it is upward closed, thus $f_{\alpha_i}(U_{\alpha_{i}}) \in\mathcal{F}$.
    
    Thus $U \in \mathcal{F}$, so $\mathcal{F} \rightarrow x$. 
\end{proof}

\begin{proposition}
    Let $X$ and $Y$ be sets, and let $f:X\rightarrow Y$ be a surjection.
    Then if $\mathcal{U}$ is an ultrafilter on $X$, then $f(\mathcal{U})$
    is an ultrafilter on $Y$.
\end{proposition}
\begin{proof}
    Suppose that $\mathcal{F}$ is some filter on $Y$ 
    that contains $f(\mathcal{U})$. Let $A \in \mathcal{F}\setminus f(\mathcal{U})$. 
    
    Let $B = f^{-1}(A)$. Since $f$ is surjective, $A = f(B)$, and $X\setminus B = f^{-1}(Y\setminus A)$.
    
    As $\mathcal{U}$ is an ultrafilter, we have either $B \in \mathcal{U}$
    or $X\setminus B \in \mathcal{U}$.
    
    If $B \in \mathcal{U}$, then since $A = f(B)$ we have $A \in f(\mathcal{U})$,
    a contradiction.
    
    Otherwise, if $X\setminus B \in \mathcal{U}$, then $Y \setminus A \in \mathcal{F}$.
    
    But then $Y \setminus A \in \mathcal{U}$. But $A \in \mathcal{U}$.
    Hence $\emptyset \in \mathcal{U}$, which is impossible.
    
    Thus $f(\mathcal{U})$ is an ultrafilter.
\end{proof}

\begin{definition}
    Let $\mathcal{F}$ be a filter on a topological space $X$. Then we
    say that $\mathcal{F}$ clusters at the point $x \in X$ if
    \begin{equation*}
        x \in \cap\{ \overline{F}\;:\;F \in \mathcal{F}\}.
    \end{equation*}
\end{definition}

\begin{proposition}
\label{clusterImpliesConverge}
    Let $\mathcal{U}$ be an ultrafilter for a space $X$.
    
    If $\mathcal{U}$ has a cluster point, then $\mathcal{U}$
    converges.
\end{proposition}
\begin{proof}
    Suppose that $\mathcal{U}$
    clusters at $x$. Let
    \begin{equation*}
        B := \{ F\cap V\;:\;F \in\mathcal{U},V \in \mathcal{N}(x)\}.
    \end{equation*}
    Then since $\mathcal{U}$ clusters at $x$, for any neighbourhood $V$
    of $x$, and any $F \in \mathcal{U}$, $V \cap F \neq \emptyset$. Hence $\emptyset \notin B$,
    so $B$ is a filter base.
    
    Let $\mathcal{F}$ be the filter generated by $B$. If $\mathcal{U}$
    does not contain $\mathcal{N}(x)$, then $\mathcal{F}$ is strictly
    larger than $\mathcal{U}$. But this is impossible, 
    so $\mathcal{U}\rightarrow x$. 
\end{proof}


\begin{proposition}
    Let $X$ be a topological space. The following are equivalent:
    \begin{enumerate}
        \item{} $X$ is compact.
        \item{} Any filter of $X$ clusters at a point.
        \item{} Any ultrafilter on $X$ converges.
    \end{enumerate}
\end{proposition}
\begin{proof}
    First we show that $(1)\Rightarrow(2)$. Since $X$ is compact, any set
    with the finite intersection property has nonempty intersection. Indeed,
    let $\mathcal{F}$ be a filter on $X$. Then clearly $\{\overline{F}\;:\;F \in\mathcal{F}\}$
    has the finite intersection property. So $\cap\{\overline{F}\;:\;F \in\mathcal{F}\} \neq \emptyset$,
    so $\mathcal{F}$ has a cluster point.
    
    Now we prove that $(2)\Leftrightarrow(3)$. Suppose that $(2)$ is true,
    then if $\mathcal{U}$ is an ultrafilter on $X$, then $X$ must have a cluster
    point. By proposition \ref{clusterImpliesConverge}, $\mathcal{U}$
    must converge.
    
    Now suppose that $(3)$ is true. Let $\mathcal{F}$ be a filter on $X$,
    and $\mathcal{F} \subseteq \mathcal{U}$
    where $\mathcal{U}$ is an ultrafilter. Then suppose that $\mathcal{U}$
    converges to $x$. Let $U$ be an open neighbourhood of $x$, and let $F \in \mathcal{F}$.
    Then we must have $U\cap F \neq \emptyset$, since $U \cap F \in \mathcal{U}$. 
    Hence $x \in \overline{F}$, so $\mathcal{F}$ clusters at $x$.
    
    Now we prove that $(2)\Rightarrow(1)$. Suppose that $\mathcal{C}$ is a cover
    for $X$ with no finite subcover. Let
    \begin{equation*}
        V := \{X \setminus \cup \mathcal{V}\;:\;\mathcal{V} \subset \mathcal{C}, |\mathcal{V}| < \infty\}.
    \end{equation*}
    Since $\mathcal{C}$ has no finite subcover, $\emptyset \notin V$. $V$ is closed
    under intersection by De Morgan's laws, so $V$ is a filterbase. Let $\mathcal{F}$
    be the filter generated by $V$, and suppose that $\mathcal{F}$ clusters at $x \in X$. 
    
    Hence $x \in X \setminus \cup \mathcal{V}$ for any finite subset $\mathcal{V}$
    of $\mathcal{C}$. But since $\mathcal{C}$ covers $X$, $X \in U$ for some $U \in \mathcal{C}$.
    But we have shown that $x \in X\setminus U$. This is a contradiction,
    hence $X$ is compact.
    
\end{proof} 

With this machinery, we can give a simple proof of the famous Tychonoff theorem:
\begin{theorem}
    Let $X = \prod_{\alpha \in A} X_\alpha$ be a product space. Then $X$
    is compact if and only if for each $\alpha \in A$, $X_\alpha$ is compact.
\end{theorem}
\begin{proof}
    Suppose that each $X_\alpha$ is compact. (The converse is easy).

    Let $\mathcal{U}$ be an ultrafilter on $X$. Then for each $\alpha$,
    the projection $\pi_\alpha(\mathcal{U})$ is an ultrafilter on $X_\alpha$
    and hence converges to a point $x_\alpha$. Hence $\mathcal{U}$
    converges to the point $(x_\alpha\;:\;\alpha\in A)$.
\end{proof}

\section{Applications to Pointless Topology}

Pointless topology is a beautiful generalisation of general topology. The idea
is that a lot of topology can be done without any reference to the points
of a space, instead one only needs the open sets. A frame is an abstraction
of the lattice of open sets on a topological space.
\begin{definition}
    A frame is a bounded lattice $(F,\vee,\wedge,0,1)$ that is closed under finite meets
    and arbitrary joins, where meets distribute over arbitrary joins.
\end{definition}
A frame is an abstract version of the lattice of open sets on a topological space.
A natural question is to ask if we can recover the points of a space from a frame.
We can do this using the concept of a \emph{completely prime filter}.
\begin{definition}
    A filter $\mathcal{F}$ on a frame $F$ is prime if $a \vee b \in \mathcal{F}$
    implies that $a \in \mathcal{F}$ or $b \in \mathcal{F}$. A filter
    is completely prime if $\bigvee_{i} a_i \in \mathcal{F}$ implies that for some $j$,
    $a_j \in \mathcal{F}$.
\end{definition}
finish this later...


\section{Applications to non-standard analysis}
When calculus was invented in the 17th century, mathematicians were
fascinated by the concept of an \emph{infinitesimal}. The idea 
was that a line can be though of as being composed of indivisible points. This
vague concept
caused a number of difficulties and paradoxes, but infinitesimals were so useful
that they were applied to great success in calculus. 

19th century mathematicians refounded calculus without infinitesimals,
since the logical difficulties were considered insurmountable.

However, using the machinery of ultrafilters, we can construct
an ordered field containing the real numbers with a plentiful
supply of infinitesimals. It is possible to do a lot of real analysis with this
field. Firstly, we introduce the idea of an ultraproduct.
\begin{definition}
    Let $\{X_i\}_{i \in I}$ be a collection of sets indexed by a set $I$. 
    Let $\mathcal{U}$ be an ultrafilter of $I$. Then define
    \begin{equation*}
        \prod_{i\in I} X_i/\mathcal{U}
    \end{equation*}
    as the set of equivalence classes in $\prod_{i \in I} X_i$, where
    $(x_i)_{i \in I}$ is equivalent to $(y_i)_{i \in I}$
    when $\{ i\;:\;x_i = y_i\} \in \mathcal{U}$.
\end{definition}

\begin{proposition}
    The definition of an ultraproduct makes sense. That is, the claimed
    equivalence relation is in fact an equivalence relation.
\end{proposition}
\begin{proof}
    Let $\sim$ be the stated relation. Clearly we have $x \sim x$
    for any $x$, since $I \in \mathcal{U}$. Similarly, the relation
    is symmetric and transitive.
\end{proof}

An ultrapower is an ultraproduct where all of the factors are the same.
\begin{definition}
    Let $\mathcal{U}$ be a non-principal ultrafilter of $\Ntrl$. Then define
    \begin{equation*}
        ^*\Rl := \Rl^{\Ntrl}/\mathcal{U}.
    \end{equation*}
    This is the set of hyperreal numbers.
\end{definition}
\end{document}
